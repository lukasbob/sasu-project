% !TEX root = ../document.tex
\section{Introduction} % (fold)
\label{sec:introduction}

\paragraph{The area of web accessibility} % (fold)
 \label{par:the_area_of_web_accessibility}
 is characterized by a stack of technologies separated from operating system accessibility frameworks. Roles, states and properties of web content and controls are mapped to the operating system's accessibility APIs for standard web controls and markup. 
 However, implementing custom controls or disregarding the semantics of HTML tags is common practice, for the purpose of distinctiveness in visual design or interactivity. A entrepreneurial environment exists on the web that wants to push the state of the art forward, sometimes employing technologies to plug some gaps in current standards-based web technologies that have detrimental effects on accessibility or usability.
% paragraph the_area_of_web_accessibility (end) 

A concurrent development is underway where public services are becoming available no-line, sometimes to the exclusion of face-to-face interaction. 
Given that this is the case, auditing web accessibility is a necessity. 
This paper will discuss a development methodology for developing auditing tools that attempts to bridge the cognitive gap between the problem space -- real-life accessibility obstacles -- and developers of web sites.

% section introduction (end)