% !TEX root = ../document.tex
\section{Discussion} % (fold)
\label{sec:discussion}

This section discusses problems with the proposed methodology; risks to its implementation; and areas that the evaluation method cannot currently address. Finally, it relates the research project to a wider field of inquiry.

\subsection{Risks, Feasibility and Shortcomings} % (fold)
\label{sub:risks_and_shortcomings_proposed_solution}

The research project represents a large undertaking, for both the researcher and the development organization. This poses a number of risks for the project and for the organization. The biggest risk is the high and continuous costs that it incurs.

Resources are required to find and interview test subjects, to run empirical usability tests, to evaluate and refine the outcomes iteratively, and to keep abreast of technological developments that may invalidate recommendations and patterns. It is perceivable that the burden of research may detract from the contributions that the researcher can make as a practitioner, and vice versa.

The outcomes of this endeavour will not translate to financial gain in anything but the long term, so it is risky for a small organization to use such a resource- and time-intensive approach. This project may be more appropriate for an organization that is not constrained by the demand of turning a profit.

However many user stories are represented, there is always a risk of under-representing or misrepresenting specific groups.

\subsection{Research Outcomes: Generalizing a Situated Process} % (fold)
\label{sub:research_outcomes_situated_theory}
As discussed, the development methodology proposed in this paper is applicable to a specific setting. This raises the question of how the outcomes of this research project relate to general theory-building within the field of software development. Checkland \cite{Checkland:1998} states as a requirement of action research that the epistemological framework according to which knowledge is to be acquired should be stated in the research design. In other words, when embarking on an action research project, the researcher should consider what constitutes knowledge, and how outcomes of the project relate to that.

In the proposed research project, the development process proposed applies very specifically to a certain class of applications. It draws upon other methodologies of software development -- Usability engineering, \acl{DDD}, Testing -- so it can draw on those same disciplines to state its outcomes. However, the central motivation for the research design is to re-frame a technocratic discussion about measuring accessibility as a humanist one about understanding and accommodating the diverse circumstances of the human condition. In more than one sense, this project is related to the anthropological field of inquiry into the concepts of justice, humanity and morality, particularly at the intersection points of cultures. Speaking of accessibility is speaking of an intersection point between cultures which may lie in a place unknown to both. It is hoped that discovering these intersection points lets us discover shared humanity.

% subsection research_outcomes_situated_theory (end)

% section discussion (end)
