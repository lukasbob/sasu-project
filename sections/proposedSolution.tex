% !TEX root = ../document.tex
\section{Proposed Development Process} % (fold)
\label{sec:proposed_development_process}

This section proposes a development methodology that facilitates increased awareness about the problem space among developers. It makes extensive use of artefacts of user research as a means of communication. It builds on the idea of \ac{DDD}, whereby a ubiquitous language permeates the development process. 

Figure \ref{fig:devprocess} illustrates the central role of artefacts in the proposed process.

\begin{figure}
\center
% !TEX root = ../document.tex
\begin{figure}
\center
\begin{tikzpicture}[node distance=1cm, auto]  
\tikzset{
    mynode/.style={rectangle,rounded corners,draw=black, top color=white, bottom color=gray!10, thick, inner sep=1em, minimum size=3em, text centered},
    myarrow/.style={->, >=latex', shorten >=1pt, thick},
    mylabel/.style={text width=7em, text centered} 
}  
\node[mynode] (users) {Users};
\node[mynode] (standards) {Standards};

\node[below=3cm of manufacturer] (dummy) {}; 
\node[mynode, left=of dummy] (retailer1) {RETAILER 1};  
\node[mynode, right=of dummy] (retailer2) {RETAILER 2};
\node[mylabel, below left=of manufacturer] (label1) {Participation rate $\theta_1$};  
\node[mylabel, below right=of manufacturer] (label2) {Participation rate $\theta_2$};
% The text width of 7em forces the text to break into two lines. 

\draw[myarrow, bend right=45] (manufacturer.south) -- ++(-.5,0) -- ++(0,-1) to (retailer1.north); 
\draw[myarrow] (manufacturer.south) -- ++(.5,0) -- ++(0,-1) -|  (retailer2.north);
% There is a slight overlap of the arrows with the (manufacturer) south edge
% because creating the offset in another way didn't compile. 
 
\draw[<->, >=latex', shorten >=2pt, shorten <=2pt, bend right=45, thick, dashed] 
    (retailer1.south) to node[auto, swap] {Competition}(retailer2.south); 
% The swap command corrects the placement of the text.

\end{tikzpicture} 


\caption{Do not forget! Make it explicit enough that readers
can figure out what you are doing.}
\end{figure}
\label{fig:devprocess}
\caption{Artefact generation forms the backbone of the development process. They act as a ubiquitous language between users, that represent the variability of the domain, and the development team. They also provide the core of the tool documentation, providing a \emph{rationale} for reported errors. Standards documents contribute to informing and codifying user stories, but their role is peripheral.}
\end{figure}

It is beyond the scope and feasibility of this proposal to promote \ac{DDD} with accessibility as a core part of the domain as a general development methodology within any organization. It is realistic that accessibility falls somewhere outside the bounded context within which the software of users of these auditing tools is developed, so it becomes the responsibility of the auditing tools to highlight errors and relay information about the accessibility domain.

\subsection{Classes of Artefacts} % (fold)
\label{sub:classes_of_artefacts}

We distinguish between two distinct types of artefacts. 

\begin{description}
	\item[User stories.] Artefacts of user research. These may be refined so that they can contribute to pedagogical material.
	\item[Case reductions.] Created iteratively by the development team based on user stories and standards documents. They serve to isolate accessibility problems, and to document patterns that work and anti-patterns that should be avoided. They can also be used to create test fixtures for automated testing in the development process.

\end{description}

% subsection classes_of_artefacts (end)

\subsection{Gathering Knowledge for User Stories} % (fold)
\label{sub:part_1_gathering_knowledge_for_user_stories}
The initial step is to gather knowledge about the many forms that accessibility challenges take. This work consists of conducting user testing with a range of users that have difficulties in accessing content on the internet, and documenting these difficulties. 
It is important to illustrate that there is always a barrier between the user and the symbols for data that the users manipulate, and that our ability to successfully interact with these symbols may be plotted somewhere on a continuum. The purpose for this is to demystify accessibility problems and to appeal to developer empathy.

% subsection part_1_gathering_knowledge_for_user_stories (end)

\subsection{Creating Artefacts that Illustrate Patterns and Anti-patterns} % (fold)
\label{sub:part_2_creating_artefacts_that_illustrate_patterns_and_anti_patterns}

The interviews with users will reveal some patterns that should be encouraged, and some anti-patterns that should be avoided. If it is possible at this point, these should be documented and codified as code snippets.

% subsection part_2_creating_artefacts_that_illustrate_patterns_and_anti_patterns (end)

% section proposed_development_process (end)