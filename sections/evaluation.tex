% !TEX root = ../document.tex

\section{Evaluation} % (fold)
\label{sec:evaluation}

The proposed solution to the problem space involves the introduction of new software development processes. The process introduces knowledge about the circumstances of the users, in the form of personas, but these personas need to be exemplified with real user stories. The process is meant to be exploratory and should encourage evolution in the way that the process occurs, and in the way that progress is measured. At the heart of this process is people, whose reactions to introduction of new processes and new knowledge will determine how the projct evolves further.

In this case, action research is an appropriate research method. Action research rejects the positivist scientific paradigm and acknowledges the mutable quality of its subject matter. Crucially, action research acknowledges that the researcher becomes part of the social practice that is enacted in the research setting.


\subsection{Software Development Processes as Social Structure} % (fold)
\label{sub:software_development_processes_as_social_structure}

A software development process is a structure designed for the express purpose of maximizing the efficiency with which high quality software is developed. Anything that detract s from this purpose is counterproductive, and should be addressed. However, software projects are also enactments of social structures. Hegemonic relationships and resistance play a part in shaping the process. 

A risk in this project is that the motivation and drive for the project does not come from a business development perspective; nor does it originate from within the development team itself. Instead it is driven by customer enquiries and dissatisfaction with the usability and comprehendability of the existing solution. 

The research setting will be shaped by the agency of the researcher, in that I act as a project lead on the project.

% subsection software_development_processes_as_social_structure (end)


\subsection{Gathering Knowledge for User Stories} % (fold)
\label{sub:part_1_gathering_knowledge_for_user_stories}

The initial step is to gather knowledge about the many forms that accessibility challenges take. This work consists of interviewing users that have difficulties in accessing content on the internet, and documenting these difficulties. It is important to illustrate that there is always a barrier between the user and the symbols for data that the users manipulate, and that our ability to successfully interact with these symbols may be plotted somewhere on a continuum. The purpose for this is to demystify accessibility problems and to appeal to developer empathy.

% subsection part_1_gathering_knowledge_for_user_stories (end)

\subsection{Creating Artefacts that Illustrate Patterns and Anti-patterns} % (fold)
\label{sub:part_2_creating_artefacts_that_illustrate_patterns_and_anti_patterns}

The interviews with users will reveal some patterns that should be encouraged, and some anti-patterns that should be avoided. If it is possible at this point, these should be documented and codified as code snippets. Listing \ref{lst:missingalt} shows this in all its glory.

\lstinputlisting[language=html, caption={Referring to a page feature by direction}, label=lst:missingalt]{code/missingalt.html}

\lstinputlisting[language=javascript, caption=A simple base class]{code/basejs.js}

% subsection part_2_creating_artefacts_that_illustrate_patterns_and_anti_patterns (end)


\begin{itemize}
	\item Participant observation
	\item 
\end{itemize}



% section evaluation (end)