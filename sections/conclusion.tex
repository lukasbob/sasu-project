% !TEX root = ../document.tex

\section{Conclusion} % (fold)
\label{sec:conclusion}

This paper has proposed an approach to developing auditing tools that uses the concrete obstacles that users face as the driver for development. The process uses incidental as well as intentionally generated artefacts to create a framework of reference that link developers and the users that they serve. The premise of the methodology is to build on the social and technical motivations that already exist within the domain.

The broader appeal of this approach is to shift the discussion about accessibility from a technocrat to a humanist endeavour: Developing accessible web content should not be framed as an exercise in risk reduction, but as a way of building community and contributing to social capital. 

The research project and the development methodology are both designed with a particular setting in mind, and for the use in development of a specific product. For this reason, wider applicability in an operational sense may not be feasible. However, the idea of emphasizing motivation and rationale over technical specification to better understand the domain ought to be applicable in many other scenarios.

% section conclusion (end)