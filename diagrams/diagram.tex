% !TEX root = ../document.tex
\begin{figure}
\center
\begin{tikzpicture}[node distance=1cm, auto]  
\tikzset{
    mynode/.style={rectangle,rounded corners,draw=black, top color=white, bottom color=gray!10, thick, inner sep=1em, minimum size=3em, text centered},
    myarrow/.style={->, >=latex', shorten >=1pt, thick},
    mylabel/.style={text width=7em, text centered} 
}  
\node[mynode] (users) {Users};
\node[mynode] (standards) {Standards};

\node[below=3cm of manufacturer] (dummy) {}; 
\node[mynode, left=of dummy] (retailer1) {RETAILER 1};  
\node[mynode, right=of dummy] (retailer2) {RETAILER 2};
\node[mylabel, below left=of manufacturer] (label1) {Participation rate $\theta_1$};  
\node[mylabel, below right=of manufacturer] (label2) {Participation rate $\theta_2$};
% The text width of 7em forces the text to break into two lines. 

\draw[myarrow, bend right=45] (manufacturer.south) -- ++(-.5,0) -- ++(0,-1) to (retailer1.north); 
\draw[myarrow] (manufacturer.south) -- ++(.5,0) -- ++(0,-1) -|  (retailer2.north);
% There is a slight overlap of the arrows with the (manufacturer) south edge
% because creating the offset in another way didn't compile. 
 
\draw[<->, >=latex', shorten >=2pt, shorten <=2pt, bend right=45, thick, dashed] 
    (retailer1.south) to node[auto, swap] {Competition}(retailer2.south); 
% The swap command corrects the placement of the text.

\end{tikzpicture} 


\caption{Do not forget! Make it explicit enough that readers
can figure out what you are doing.}
\end{figure}